\documentclass{article}
\usepackage[utf8]{inputenc}
\usepackage[letterpaper, margin=1.5in]{geometry}
%\usepackage{times}
\usepackage{natbib}
\usepackage{gb4e}
\begin{document}

\title{Verb-Related Errors Made by Japanese Learners of English}
\author{Derek Andersen \\ \\
        Dept. of Linguistics \\
		Stony Brook University \\
		\texttt{derek.andersen@stonybrook.edu}} 
\date{}
\maketitle

\section{Introduction}
Japanese, a member of the Japonic family of languages, is spoken by roughly 125 million people worldwide, and mainly in Japan. Students in Japan are required to start learning English from the age of ten. Native speakers of Japanese who are learning English commonly make two general types of English errors: those which are a result of L1 interference, or the influence of their native language (Type 1), and those which are simply the result of misunderstanding while learning, or over-generalization of certain English grammatical principles (Type 2) \citep{Bryant1984}. In this paper, I will highlight two different Type 1 errors and discuss their implications about the contrast between English and Japanese grammar. The focus will be on verb-related errors made by Japanese learners of English -- namely, errors with verb inflection and errors with verbal aspect. In Section 2, I will present a common error made by Japanese English language learners (hereafter ELLs) regarding subject-verb agreement in English, and discuss why the nature of agreement in English might cause difficulty for Japanese ELLs. In Section 3, I will present another common error made by Japanese ELLs regarding verbal aspect, and discuss how verbal aspect differs in English and Japanese. In Section 4, I conclude that the errors made by Japanese ELLs presented in this paper support the idea that interference of the mother tongue of a Japanese ELL has a significant effect on English acquisition.

\section{Verb Inflection Errors}

In English, main verbs are inflected to show person and number, as with the third person singular suffixes \textit{-s} and \textit{-es} for example, shown in (\ref{3rdsg}).

\begin{exe}
	\ex
    \label{3rdsg}
    	\begin{itemize}
    		\item[a] “John walk\textit{s} the dog."
            \item[b] “Mary go\textit{es} to the store."
    	\end{itemize}
\end{exe}
In contrast, Japanese does not use inflection on the main verb to show person and number \citep{Bryant1984}. Instead, for a sentence like (\ref{3rdsg}b) which is in the simple present form, the Japanese counterpart would use the verb \textit{iku} ‘to go’ in its uninflected form, similar to the English bare infinitive ‘go’.  For this reason, native Japanese speakers learning English commonly make verb inflection errors, as shown in (\ref{error1}). This can be seen as an effect of their mother tongue on their English acquisition.
\begin{exe}
	\ex[*]{“Mary go to the store."}\label{error1}
\end{exe}
To go further with this issue of subject-verb agreement in Japanese ELLs, we can look at third person singular more closely. In a case study referenced by \cite{O'Grady2006} which involved six native speakers of Japanese and focused on their English acquisition, the speakers correctly used the third person singular \textit{-s} suffix less than 20\% of the time. From this, it’s apparent that the acquisition of subject-verb agreement in ELLs (at least, those who are native speakers of a language like Japanese in which this phenomenon is absent) is quite difficult.

\subsection{Analysis of Verb Inflection Errors}

In order to possibly pinpoint the difficulties in the acquisition of subject-verb agreement, first the nature of agreement in language must be analyzed. As pointed out by \cite{O'Grady2006}, “in pedagogical grammar and even in discussions of second language acquisition, agreement is treated as a simple factual matter”. In other words, the approach to teaching English to non-native speakers is to regard subject-verb agreement as something akin to the meaning of lexical items, or other learned facts, in that it can be accessed from the mind (during speech) as quickly as these other facts. However, as proposed earlier, in the building and refining of an ELL's grammar, the mother tongue of the speaker (in this case, Japanese) inevitably intrudes and affects the acquisition of certain grammatical constructions. For this reason, it might be wrong to regard the phenomenon of agreement as factual or easily taught in a classroom setting. Rather, it can be looked at as more procedural – something that requires computational, step-by-step sentence-building operations in the mind of the speaker \citep{O'Grady2006}.\\
	
    Evidence for this procedural view can be seen in (\ref{agreement}). Contrastive to (\ref{error1}), in which the main verb of the sentence needs to agree with the subject, (\ref{agreement}) shows examples of the main verb agreeing with an NP other than the subject – specifically, a direct object. In these examples, the inclusion of the empty subject \textit{there} switches the agreement from subject-verb to object-verb, further complicating things for the ELL.
\begin{exe}
	\ex
    \label{agreement}
    	\begin{itemize}
        	\item[a]
            	\begin{tabular}{c c c c c c c}
            	“There & is & a & book & on & the & table." \\
            	{} & \{Verb\} & {} & \{Object\} & {} & {} & {} 
            	\end{tabular}
            \item[b]
            	\begin{tabular}{c c c c c c c}
                “There & are & books & on & the & table." \\
                {} & \{Verb\} & \{Object\} & {} & {} & {}
                \end{tabular}
        \end{itemize}
\end{exe}
The reason (\ref{agreement}) supports the procedural view of agreement has to do with the method by which agreement is processed and satisfied by speakers. In a sentence like (\ref{3rdsg}a) or (\ref{3rdsg}b), where subject and verb are the constituents that are needed for the agreement to be satisfied, the sentence is processed from left to right by the speaker in a piecewise fashion \citep{O'Grady2006}. This is exemplified in the flowchart in (\ref{flowchart}), which illustrates the process for sentence (\ref{3rdsg}b) (recalled below).

\newpage

\begin{itemize}
	\item[(\ref{3rdsg}b)]{“Mary goes to the store."}
\end{itemize}

\begin{exe}
	\ex{Subject-verb agreement flowchart for sentence (\ref{3rdsg}b)}
    \label{flowchart}
    	\begin{itemize}
        	\item[i]{‘Mary', the first word of the sentence, satisfies the need for a subject}
            \item[ii]{‘go', the following word, satisfies the need for a main verb}
            \item[iii]{The main verb is inflected with respect to the subject's person/number (and becomes ‘goes')}
            \item[iv]{Subject-verb agreement is satisfied}
        \end{itemize}
\end{exe}
A sentence like (\ref{agreement}a) or (\ref{agreement}b), however, might complicate this process for a Japanese ELL. When approached with a similar method as above, the sentences in (\ref{agreement}) can understandably cause problems for Japanese ELLs. This is because of the empty subject \textit{there} at the beginning of the sentences. While it is the subject of the sentence, it doesn’t have control over the inflection of the verb, contrary to what an ELL might think based on the method used in (\ref{flowchart}). Instead, the presence of the empty subject signals that the direct object, \textit{a book}, is the item with which the main verb needs to agree. Crucially, this switch of focus to the direct object here can be attributed to the empty subject’s lack of both person and number values – the key components needed for agreement to take place.

\section{Verbal Aspect Errors}

Verbal aspect, or the information used to express how the speaker views the action of a verb, is another commonly problematic area of English grammar for Japanese ELLs. The verb inflection errors discussed in the previous section were prefaced with the idea that English has the notion of subject-verb agreement, while Japanese does not. Verbal aspect, however, is a grammatical component of both English and Japanese. So, what accounts for these errors in usage? In this section I highlight the English verbal aspect errors made by Japanese ELLs, why they might occur contrary to the existence of aspect in Japanese, and how these errors illustrate differences between English and Japanese grammar.
\begin{exe}
	\ex[*]{“Each things in the film were showing the characteristics of the scene very well.”}\label{error2}
\end{exe}
Sentence (\ref{error2}) \citep{Bryant1984} exemplifies a common type of sentence erroneously produced by Japanese ELLs. What’s relevant here to our analysis of aspectual errors is the form of the main verb: \textit{were showing}. Here, the speaker has incorrectly used the past progressive form of the verb, where the simple past form should have been used. Importantly, the tense of the verb chosen by the Japanese ELL was correct, but the aspect chosen was incorrect. From this we can assume that aspect poses an issue for the learner. However, as stated earlier, Japanese and English both have verbal aspect as part of their grammars – thus the problem must not lie in the learner’s inability to understand the notion of verbal aspect, but rather in the differences between verbal aspect (or at least, the progressive aspect) in English and Japanese.

\newpage

\subsection{Analysis of Verbal Aspect Errors}

In English, progressive aspect is generally used to express actions that are ongoing. While it can be used in conjunction with the past tense, i.e. the past progressive form, this is used for an activity continuing in the past – to indicate a time span in the past, italicized in (\ref{pastprog}a). For comparison, the simple past form is italicized in (\ref{pastprog}b).

\begin{exe}
	\ex
    \label{pastprog}
    \begin{itemize}
    	\item[a]{“While I \textit{was doing} my work, he called me.”}
        \item[b]{“After I \textit{did} my work, he called me."}
    \end{itemize}
\end{exe}
For this reason, the use of the past progressive form in (\ref{error2}) is confusing to a native English ear. Any native speaker of English would use the simple past form here, simply because the main interest in (\ref{error2}) is expressing what the things in the film did, rather than the duration of the action. But the choice of the past progressive form in (\ref{error2}) by the Japanese ELL isn’t completely surprising. Japanese also uses progressive aspect; for example, \textit{itte iru} ‘he is going’ or \textit{tabete iru} ‘he is eating’. However, the situations in which the progressive aspect is used in the two languages vary \citep{Bryant1984}.\\

Why, though, does the progressive construction in (\ref{error2}) sound unnatural to any native speaker of English? The answer lies in the class of the main verb, \textit{show}. Traditionally, verbs can be separated into two lexical classes: stative and dynamic. While stative verbs express states or situations, dynamic verbs represent actions. In English, stative verbs normally do not occur in progressive aspect forms (e.g. *“I \textit{am knowing} the answer to the question.”) \citep{Cowan2008}. Since the verb in (\ref{error2}), \textit{show}, is stative in that it is used to denote something that the things in the film inherently do, the progressive aspect would not be used.\\

In order to understand why a Japanese ELL might erroneously choose to use the progressive aspect in this situation, we can extend the idea of stative and dynamic verbs into our analysis of Japanese grammar. As we’ve seen, English does not normally allow the use of stative verbs in progressive aspect forms; only dynamic verbs like \textit{eat} and \textit{walk} occur in progressive constructions regularly. Japanese, on the other hand, allows the use of both dynamic verbs and stative verbs in its progressive aspect constructions. According to \cite{Bryant1984}, Japanese allows not only the use of mental activity-denoting verbs (like \textit{love} and \textit{know}) in progressive aspect constructions, but also sensation- or perception-denoting verbs (like \textit{hear} and \textit{see}). Both of these categories fall under the heading ‘stative verbs’, and thus aren’t normally allowed to occur in progressive aspect constructions in English.

\section{Conclusion}
While English and Japanese grammar are alike in certain aspects, like the existence of verbal aspect, they differ in others. I’ve presented some common verb-related errors made by Japanese ELLs that illustrate two important differences between English and Japanese grammar – namely, the absence of subject-verb agreement in Japanese versus its presence in English, and the differing scenarios in which the progressive verbal aspect can be used in the two languages. Not only did the errors and analyses presented highlight the differences between the grammars of the two languages, but they also supported the idea that interference of the mother tongue of a Japanese ELL has a significant effect on English acquisition.

\bibliographystyle{apa}
\bibliography{refs}

\end{document}
