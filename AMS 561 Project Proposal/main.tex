\documentclass{article}
\usepackage[utf8]{inputenc}

\title{AMS 561 Project Proposal}
\author{Joanne Chau and Derek Andersen}
\date{April 7, 2020}

%No page numbering
\pagenumbering{gobble}

\begin{document}

\maketitle

\section*{Topic}

The title of our project is \textit{Implementing First Order Logic in Python to Evaluate Phonological Phenomena}.

\section*{Significance}

This topic is significant to us because we are both Computational Linguistics majors. We are interested in applying the techniques and strategies from this class to our studies and research. Computational Phonology is a field that has yet to be explored in the computer science world through the use of logical languages like first order logic and monadic second order logic.  

\section*{Outcomes}

We hope to build a program that evaluates input and output of strings to give the user the phonological processes the string underwent from the input to the output. In phonology, it is theorized that every speakers' brain accepts an input and has constraints against how things are said (output), which is why the input and output are often different. For example, plurals in English are pronounced differently based off of the sound proceeding the plural marker. In recent years, phonologists have been attempting to use logical languages to explain these constraints. But this has not been explored much in computer science, and we hope to push the boundaries for further developments within this field.

\section*{Techniques}

We will be using Python as our main programming language as well as other techniques that were explored during this class. We will also be using our understanding of phonology in the development of this program. We have both taken several phonology classes throughout the course of our graduate studies, and we hope that our knowledge and experience on the topic will help us to implement this program as best we are able.

\end{document}
